\begin{figure}[htbp]
\centering
\includegraphics[width=4in]{mmm-traits.pdf}
\caption{
Desirable traits of ecological metaphors, metrics, and management approaches
(decision-making tools).
}
\label{fig:traits}
\end{figure}

\clearpage

\begin{figure}[htbp]
\centering
\includegraphics[width=3in]{efficient-frontier-fig.pdf}
\caption{
An introduction to Modern Portfolio Theory mean-variance optimization. In
finance, portfolios are formed by choosing how much to invest in various
assets. Modern Portfolio Theory focuses on identifying the set of portfolios
that optimizes the trade-off between expected return (mean) and expected
variance or risk. (a) This set of portfolios is referred to as the efficient
frontier. (b) The minimum variance portfolio achieves the lowest expected
risk; the remaining risk is said to be undiversifiable. (c) A risker, but
still efficient portfolio. (d) An example inefficient portfolio, which has a
lower expected return than (c) and greater expected risk than (b). Adapted
from \citet{hoekstra2012}.
}
\label{fig:mpt}
\end{figure}

\clearpage

\begin{figure}[htbp]
\centering
\includegraphics[width=4.2in]{salmon-portfolios-bw.pdf}
\caption{
There are multiple ways of investing in ecological portfolios. In this
example, investors are shown along the top and bottom and ecological assets
are shown in the middle (populations of chum salmon, \emph{Oncorhynchus keta},
and Chinook salmon, \emph{Oncorhynchus tshawytscha}). The shaded arcs indicate
investment. (a) Partial-asset investors invest by removing portions of the
salmon populations --- the salmon that commercial fisher 1 removes are
unavailable for the grizzly bear. These investors can often change their
investment with ease. For example, commercial fisher 2 could decide to fish
more Chinook and less chum salmon. Most financial portfolio theory is
developed around this paradigm. (b) Whole-asset investors invest in entire
populations. These investors can share assets but may have different goals for
their portfolio. They can adjust their investment by managing properties of
the population itself. For example, the fisheries management agency could
reduce fishing of chum salmon to allow the population to grow. The
conservation agency could fund habitat restoration for Chinook salmon to
increase carrying capacity and expand their investment.
}
\label{fig:salmonport}
\end{figure}

\begin{figure}[htbp]
\centering
\includegraphics[width=5.5in]{risk-fig/risk-fig-bw.pdf}
\caption{
Symmetric variability vs.~downside risk metrics. (a--c) An illustration of
three theoretical systems with the same symmetric variability but different
levels of downside risk. The y-axis denotes rate of change of abundance or
biomass (``returns'' in financial terminology). The dashed lines represent the
mean ($\mu$) and the surrounding shaded regions represent $\pm$ one standard
deviation ($\sigma$) --- a measure that does not account for the asymmetric
property of risk. The solid lines represent the 95\% CVaR (conditional value
at risk) and the surrounding shaded regions represent the regions below the
95\% VaR (value at risk). CVaR and VaR are both downside risk metrics that can
accurately identify higher-risk systems. (d, e) Symmetric and downside risk
metrics applied to annual returns of sockeye salmon (\emph{Oncorhynchus
  nerka}) stocks in the Fraser River (\citep[data from][]{dorner2008}). Metric
values are scaled to a maximum of 1.0 across stocks and stocks are ordered by
increasing CVaR. Symmetric variability (CV; CV = $\sigma / \mu$) and
asymmetric risk metrics (all others) differ considerably in their rank order
of risk for the stocks.
}
\label{fig:risk}
\end{figure}
