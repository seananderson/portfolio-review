\LTcapwidth=\textwidth
\bibpunct{}{}{;}{a}{}{;}
%\singlespacing
\begin{footnotesize}

\begin{longtable}{>{\RaggedRight}p{2.7cm}>{\RaggedRight}p{3.3cm}>{\RaggedRight}p{3.9cm}>{\RaggedRight}p{4.2cm}}

\caption{Attributes of financial and ecological data and their implications for ecological portfolios.}\\
\toprule

\textbf{Data attributes} &
\textbf{Financial portfolios} &
\textbf{Ecological portfolios} &
\textbf{What this means for ecological portfolios}\\

\midrule

Interdependence &
A diverse portfolio is unlikely to have strong dependence between assets &
Populations may be impacted by changes in other populations of the same species (e.g.\ competition or migration) or other species (e.g.\ predation) &
Interactions need to be accounted for; forecasts may be less reliable\\

Measurement error &
Reported asset value is the true value that impacts an investor &
Recorded data typically includes substantial measurement error and may be biased &
Uncertainty needs to be propagated through analyses\\

Frequency and duration &
High frequency (e.g.\ seconds), often regular recording intervals, missing values rare, long durations &
Lower frequency (often years), often irregular recording intervals, missing values common, often short durations &
Greater uncertainty in optimal solutions; time-series methods require unique approach (e.g.\ missing values may need to be imputed, autocorrelation different)\\

Synchrony &
Relatively low synchrony for diversified portfolios &
Relatively high synchrony for similar populations or species; potential asynchrony due to species interactions &
Potentially less benefit from portfolio diversification; species interactions may alter portfolio dynamics\\

Mean-variance scaling &
Variance scales directly with investment &
Variance may scale indirectly with investment if asset size itself is considered an investment weight &
The mean-variance relationship may need to be accounted for (\citeauthor{anderson2013} \citeyear{anderson2013})\\

Number of assets &
Typically unlimited; generally high &
Typically limited; generally low &
Potentially less opportunity for portfolio diversification\\

\bottomrule
\label{tab:data}
\end{longtable}
\end{footnotesize}
